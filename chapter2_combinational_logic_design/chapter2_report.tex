\documentclass[12pt]{article}

\usepackage{amsmath}
\usepackage{caption}
\usepackage[margin=1in]{geometry}
\usepackage{graphicx}
\usepackage[colorlinks=true, allcolors=blue]{hyperref}
\usepackage[utf8]{inputenc}
\usepackage[section]{placeins}

\graphicspath{{./figures}}

\title{ECE 271 - Chapter 2 Reading Report}
\author{Phi Luu}
\date{\today}

\begin{document}

\maketitle

\section{Chapter Outline}

This chapter covers the basics of combinational logic design and heavily focuses on the functional and timing relationships between inputs and outputs of a circuit. In this chapter, the authors use boolean and boolean algebra to establish multilevel combinational logic and help reduce the amount of hardware to build more complex circuits. Another focus of this chapter is showing how to use Karnaugh maps (or K-maps) to minimize logic in a more graphical and intuitive way. Each of the topics mentioned will be discussed further throughout the following sections.

\begin{enumerate}
    \item \textbf{Introduction}
    \item Boolean Equations
    \item Boolean Algebra
    \item From Logic to Gates
    \item Multilevel Combinational Logic
    \item X's and Z's, Oh My
    \item Karnaugh Maps
    \item Combinational Building Blocks
    \item Timing
    \item Summary
\end{enumerate}

\section{Grey Box Exploration}

\section{Figures}

\section{Example Problems}

\section{Glossary}

\section{Interview Question}

\section{Reflection}

\section{Questions for Lecture}

\bibliographystyle{ieeetr}
\bibliography{references}

\end{document}

