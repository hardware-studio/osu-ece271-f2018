\documentclass[12pt]{article}

\usepackage{amsmath}
\usepackage{array}
\usepackage{caption}
\usepackage[top=1in, bottom=1in, left=0.75in, right=0.75in]{geometry}
\usepackage{graphicx}
\usepackage[colorlinks=true, allcolors=blue]{hyperref}
\usepackage[utf8]{inputenc}
\usepackage{multirow}
\usepackage{pdfpages}
\usepackage[section]{placeins}

\graphicspath{{figures/}}

\begin{document}

\begin{titlepage}
  \begin{center} \LARGE
    \vspace*{1.5in}

    ECE 272 Pre-Lab 4

    Fall 2018

    \vfill

    System Verilog and Complex Projects

    Phi Luu

    \vfill

    October 24\textsuperscript{th}, 2018

    Grading TA: Edgar Perez

    \vspace{1.5in}
  \end{center}
\end{titlepage}

Lab 4 is the first lab implemented with a hardware description language. For this class, we will be using System Verilog. Take a look at chapter 4 of the textbook and answer the following questions:

\begin{enumerate}
  \item In your own words, what is a module?

  A module is a standalone subsystem with inputs, outputs, and internal structures. It acts like a black box and can connect with other modules to make up a grand system that servers particular purposes.

  \item What is a Bus? How do you designate one is System Verilog?

  A Bus is a single wire running across a common signal of all---for example, modules---and connecting each of the signal into that wire. I think of the bus seats as the common signal of the modules and the bus floor as the Bus wire connecting all signals into a single Bus signal.

  In Verilog, a Bus can be designated by being connected to multiple signal across the system.

  \item What does the term Logic mean in System Verilog?

  A \textit{Logic} in Verilog can be used either as a \textit{wire} or as a \textit{register}. Logic blocks are generally busses, and the way they are used determines whether they are wires or registers.
\end{enumerate}

\end{document}
